\chapter{Einleitung}
\label{einleitung}
Idee & Vision

\section{Ziel}
Finde Geheimtipps für lokale Cafés im Umkreis
Lerne neue Kaffeesorten und Zubereitungsarten kennen

\section{USP}
Anzeige lokaler Cafés und damit nachhaltige Unterstützung der lokalen Kaffeewirtschaft gegen die großen Ketten



\chapter{Konzeption}
\label{konzeption}
Personas Screenshots Prototyp, Personas wie in der Präsi, Flow Chart, Anforderungsanalyse, Design

\section{Zielgruppe}
Primäre Zielgruppe
Kaffeeliebhaber, z.B. Studenten, Arbeitnehmer
„Support your locals“-Mentalität

Sekundäre Zielgruppe
Urlauber
Touristen

\section{Personas}
\subsection{Primärpersona}
Person
Laura Wagner
25 Jahre, ledig
Studium Sozialpädagogik
Smartphone: Samsung Galaxy A5 (Android)
Mentalität
Extrovertiert, spontan
Geht gerne mit Freunden im kleinen Café um die Ecke was trinken, will aber auch andere Cafés entdecken
Legt Wert auf Fairtrade- und Biowaren, gemütliche und individuelle Cafés, kurze Wege
Technikaffin (Mobil, Social Media)

\subsection{Sekundärpersona}
Person
Roland Reisenweber
42 Jahre, verheiratet
Journalist bei Frankfurter Allgmeine
Weniger technikaffin, iPhone
Mentalität
Extrovertiert, offen für neues
Arbeitet gerne an öffentlichen Orten
Soziales Bewusstsein
Ist gerne mit seiner Frau auf Reisen (innerhalb der EU)
Will regionale Kulturen kennenlernen


\section{Anforderungsanalyse}
\subsection{Muss-Anforderungen}
Muss-Anforderungen auf das Nötigste heruntergebrochen, um die App sinnvoll bedienen zu können

Lokale Cafés in der Nähe anzeigen
Kartenansicht in Google Maps
Listenansicht
Standortdaten via Google API
Kaffeelexikon
Einträge über Zubereitungsarten
Schüttelfunktion für Zufallswahl einer Zubereitungsart

\subsection{Soll-Anforderungen}
Navigation zu ausgewähltem Café über Google Maps
Listenansicht mit den wichtigsten Standortinformationen
Detailseite mit mehr Infos über das Café (Mehr Infos: Abhängig von Google API)

\subsection{Kann-Anforderungen}
Mehr Daten über zusätzliche APIs
Eigene Standortbewertungen durch die App
Eigene Standorte vorschlagen
Erweitertes Lexikon
Profil

=> Anforderungen formulieren, Anforderungen in kleine Pakete unterteilen


\section{Ablaufplan}
=> Aufwandsschätzung schwierig (Erfahrung fehlt)
Erfahrung: React Native, Cross Plattform, allg. App-Entwicklung

\section{Flow Chart}
\section{Wireframes}
\section{Mockups}
\section{Prototyp}
Mockups aus Wireframes
Interaktiver Prototyp aus Flowchart und Mockups




\chapter{Entwicklung}
\label{entwicklung}


\section{React Native}
Das technische Kernelement des Projekts stellt das \emph{React Native} Framework dar.
Als ein von \emph{Facebook} entwickeltes und verwaltetes Repository erfreut es sich mit über 58.000 Sternen auf \emph{GitHub} großer Beliebtheit.
Demnach steht bereits eine große Community hinter dem Cross-Plattform-Framework.
Die Entwicklung mit React Native erfolgt fast ausschließlich über JavaScript.
Als Cross-Plattform-Framework ermöglicht React Native den Zugriff auf native Komponenten via JavaScript und profitiert somit von einer besseren Performance als hybride Frameworks.
Allerdings erscheint es trotz seiner Popularität technisch noch nicht sehr etabliert.
Darüber hinaus benötigen die nativen Komponenten meist noch Anpassungen per Hand für Android und iOS.


\section{Setup}
React Native: Quick Start, Native Code: create-react-native-app
npm / yarn
Node.js
Android Studio
Xcode
Visual Studio Code als Editor
Simulator oder Smartphone
Git \& Bitbucket
Trello

=> Viele Tools für kleine App, Projektmanagement-tool zu aufwändig für Projektgröße



\section{Arbeitsaufteilung}
Zusammen angefangen

\subsection{Gemeinsam}
Grundgerüst, Navigation, Karte
Grundgerüst gebaut mit Navigation und Seitenstruktur
Anschließend: Jeder in andere Richtung weiter gearbeitet
Karte: Aufteilung in Android und iOS
Undokumentierte Module: Google Maps Karte

=> Tabs und Bottom Navigation, Undokumentierte Module aus der Community


\subsection{Katharina}
Lexikon, Detailseite Kaffee, Bewegungssensor

=> Integration Bewegungssensor und Navigation, Native UI Komponenten

\subsection{Thomas}
Geolocation, Liste, Detailansicht Café
Google Places API
Google Maps API

=> API-Anbindung, Native UI Komponenten



\section{Hands-on}
Praktischer Teil, was lief gut, was nicht

\subsection{Entwicklungsumgebung}
VS Code
Simulatoren
Einbindung in Android Studio \& XCode



!!! Learnings: hier passend, aber anders formulieren !!!
Brücke zwischen Web- und mobiler Entwicklung: Brücke zwischen Web- und mobiler Entwicklung (einfacher Einstieg)
Umständliches Debugging (Simulator -> kein CSS debugging)
Native Abhängigkeiten trotz Cross-Plattform (Libraries einbinden, Environment konfigurieren)
Viele aber unausgereifte Module zur Verfügung – nicht immer ganz ausgereift oder gut dokumentiert


\subsection{Module}
Map, Accelerator



\section{Vorstellung der App}
Screenshots: Startseite, Map, List, Detail Café, Bewertungen, Lexikon, Alert, Lexikon Detail

??? Beide Plattformen ???


\section{Readme}






\chapter{Ausblick}
\label{ausblick}
...


\section{Testkonzept}

Usability Test
10 – 20 Versuchsteilnehmer aus der Zielgruppe
Beobachtung
Thinking Aloud

Interview und A/B-Test als Teil des Usability-Tests


Interview
Halb-strukturiert
offene Fragen -> allgemein zur App
geschlossene Fragen ->   A/B-Test

A/B-Test
Analyse Schüttelfunktion
Vorschlag Kaffeesorte
Bezahlfunktion



\section{Marketing}
Produktpolitik
App
Web-App
Preispolitik
Kostenfreier Download
Eigenes Café als Top-Café (Anzeige)
Userdaten / In-App Werbung?
Cafédaten
Distributionspolitik
App-Stores
Browser
Kommunikationspolitik
Landing-Page / Social Pages
Werbung in Cafés
Influencer
Social Media Werbung
Zeitung
Blog
Coffee of the Week, Location of the Week, Kaffee-Trends
Kaffee-Reporter (Gesicht für die App)
"Support your Locals" - Communities

Retention Marketing
Gutschein für das Café durch die App
Cafébesitzer müssen zustimmen
Gamification
Punkte bei Besuch neuer Cafés
Top10 bei jeweiligem Café
Auf den Seltenheitsgrad von Kaffeesorten aufmerksam machen
Kaffee-Trends
Lexikoneinträge durch Cafébesuch freischalten
Zeitliche Limitierung

=> Marketing: Eigenes Themengebiet mit vielen Facetten






\chapter{Fazit}
\label{fazit}


\section{Lessons Learned}
Erste Schritte mit React Native und Cross-Plattform-Entwicklung
Umgang mit APIs
Umsetzung der nativen Designs
Ansprechen von Hardwareschnittstellen
Herausforderungen bei der Benutzung weniger etablierter Frameworks
Technische Konzeption: Anforderungen und Implementierung <-> Produktkonzeption (König): Zielgruppe, Marktanlyse, etc.




\section{React Native}
Brücke zwischen Web- und mobiler Entwicklung: Brücke zwischen Web- und mobiler Entwicklung (einfacher Einstieg)
Umständliches Debugging (Simulator -> kein CSS debugging)
Native Abhängigkeiten trotz Cross-Plattform (Libraries einbinden, Environment konfigurieren)
Viele aber unausgereifte Module zur Verfügung – nicht immer ganz ausgereift oder gut dokumentiert


\section{Cross-Plattform vs. Nativ}
Angelehnt an Webentwicklung
Gute Designmöglichkeiten
Hardwarezugriff
Für ComfyCoffee nativ zu aufwändig

=> Cross-Plattform Entwicklung lieber als native Entwicklung