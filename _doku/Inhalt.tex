\chapter{Einleitung}
\label{einleitung}
Idee & Vision

\section{Ziel}
Finde Geheimtipps für lokale Cafés im Umkreis
Lerne neue Kaffeesorten und Zubereitungsarten kennen

\section{USP}
Anzeige lokaler Cafés und damit nachhaltige Unterstützung der lokalen Kaffeewirtschaft gegen die großen Ketten



\chapter{Konzeption}
\label{konzeption}
Personas Screenshots Prototyp, Personas wie in der Präsi, Flow Chart, Anforderungsanalyse, Design

\section{Zielgruppe}
Primäre Zielgruppe
Kaffeeliebhaber, z.B. Studenten, Arbeitnehmer
„Support your locals“-Mentalität

Sekundäre Zielgruppe
Urlauber
Touristen

\section{Personas}
\subsection{Primärpersona}
Person
Laura Wagner
25 Jahre, ledig
Studium Sozialpädagogik
Smartphone: Samsung Galaxy A5 (Android)
Mentalität
Extrovertiert, spontan
Geht gerne mit Freunden im kleinen Café um die Ecke was trinken, will aber auch andere Cafés entdecken
Legt Wert auf Fairtrade- und Biowaren, gemütliche und individuelle Cafés, kurze Wege
Technikaffin (Mobil, Social Media)

\subsection{Sekundärpersona}
Person
Roland Reisenweber
42 Jahre, verheiratet
Journalist bei Frankfurter Allgmeine
Weniger technikaffin, iPhone
Mentalität
Extrovertiert, offen für neues
Arbeitet gerne an öffentlichen Orten
Soziales Bewusstsein
Ist gerne mit seiner Frau auf Reisen (innerhalb der EU)
Will regionale Kulturen kennenlernen


\section{Anforderungsanalyse}
\subsection{Muss-Anforderungen}
Muss-Anforderungen auf das Nötigste heruntergebrochen, um die App sinnvoll bedienen zu können

Lokale Cafés in der Nähe anzeigen
Kartenansicht in Google Maps
Listenansicht
Standortdaten via Google API
Kaffeelexikon
Einträge über Zubereitungsarten
Schüttelfunktion für Zufallswahl einer Zubereitungsart

\subsection{Soll-Anforderungen}
Navigation zu ausgewähltem Café über Google Maps
Listenansicht mit den wichtigsten Standortinformationen
Detailseite mit mehr Infos über das Café (Mehr Infos: Abhängig von Google API)

\subsection{Kann-Anforderungen}
Mehr Daten über zusätzliche APIs
Eigene Standortbewertungen durch die App
Eigene Standorte vorschlagen
Erweitertes Lexikon
Profil

=> Anforderungen formulieren, Anforderungen in kleine Pakete unterteilen


\section{Ablaufplan}
=> Aufwandsschätzung schwierig (Erfahrung fehlt)
Erfahrung: React Native, Cross Plattform, allg. App-Entwicklung

\section{Flow Chart}
\section{Wireframes}
\section{Mockups}
\section{Prototyp}
Mockups aus Wireframes
Interaktiver Prototyp aus Flowchart und Mockups




\chapter{Entwicklung}
\label{entwicklung}


\section{React Native}
Nach kurzer Diskussion und dem Abwägen von Vor- und Nachteilen entschieden wir uns das \emph{React Native} Framework zu benutzen.
React Native stellt somit das technische Kernelement des Projekts dar.
Als ein von \emph{Facebook} entwickeltes und verwaltetes Repository erfreut es sich mit über 58.000 Sternen auf \emph{GitHub} großer Beliebtheit.
Demnach steht bereits eine große Community hinter dem Cross-Plattform-Framework.
Die Entwicklung mit React Native erfolgt fast ausschließlich über JavaScript.
Als Cross-Plattform-Framework ermöglicht React Native den Zugriff auf native Komponenten via JavaScript und profitiert somit von einer besseren Performance als hybride Frameworks.
Allerdings erscheint es trotz seiner Popularität technisch noch nicht sehr etabliert.
Darüber hinaus benötigen die nativen Komponenten meist noch Anpassungen per Hand für Android und iOS.


\section{Setup}
Um React Native einem Projekt hinzuzufügen stehen zwei Möglichkeiten zur Verfügung.
Eine davon ist der \emph{Quick Start}, welcher nach erfolgreichem Ausführen einen QR-Code anzeigt.
Über das Smartphone kann der Code gescannt und die App aufgerufen werden.
Vor- und Nachteil gleichermaßen ist, dass die Entwicklungsumgebungen \emph{Android Studio} für Android und \emph{Xcode} für iOS nicht installiert sein müssen.
Allerdings ist es somit auch nicht möglich, selbsterstellte React Native Module hinzuzufügen.
Demnach entschieden wir uns für die zweite Variante, um zusätzlich Zugriff auf eigene Module zu haben.
Hierbei wird React Native über die Kommandozeile und den Befehl 'create-react-native-app' initialisiert.
\\
Grundvoraussetzungen hierbei sind eine funktionierende Version von \emph{Node.js} sowie das Facebook Tool \emph{Watchman}.
Für die Verwaltung von Abhängikeiten wird bei der Installation von React Native der Paketmanager \emph{Yarn} hinzugefügt, der Zugriff auf die über den \emph{Node Package Manager (npm)} bereitgestellten Module ermöglicht.
Zusätzlich müssen die beiden Entwicklungsumgebungen Android Studio und Xcode zum Kompilieren des nativen Codes bereit stehen.
Testing und Debugging der App kann somit entweder über die in Android Studio und Xcode integrierten Simulatoren erfolgen oder aber direkt über ein verbundenes Smartphone.
\\
Unsere Wahl für einen Code Editor fiel auf \emph{Visual Studio Code} von Microsoft, da es einen einfachen Einstieg bietet und sowohl für Windows als auch für Mac verfügbar ist.
Als Versionierungssystem benutzten wir Bitbucket, welches die Verwaltung von Git-Repositories ermöglicht.
Des Weiteren richteten wir ein \emph{Trello-Board} ein, auf dem das Projektmanagement abgebildeten werden sollte.
\\
\underline{Learnings:}
Obwohl die Konzeption unserer App auf einen eher geringen Umfang schließen lässt wurden doch sehr viele Tools benötigt, um eine Entwicklung zu ermöglichen.
Für diese Projektgröße stellte sich auch das Projektmanagementtool Trello als zu aufwändig in der Pflege heraus.



\section{Arbeitsaufteilung}
Die Aufteilung der anstehenden Aufgaben sahen wir als notwendig an, um die geplanten Deadlines einhalten zu können.
Unumgänglich war allerdings eine gemeinsame Code-Basis, in die zusätzliche Funktionalitäten unabhängig voneinander integriert werden konnten.
Hierzu legten wir durch \emph{Pair Programming} zusammen eine geeignete Grundlage.

\subsection{Gemeinsam}
Das Fundament besteht aus dem Grundgerüst der App sowie der Navigation durch diese.
Dabei besteht das Navigationskonzept aus einer übergeordneten \emph{Stack Navigation} sowie mehreren kontextabhängigen, verschachtelten \emph{Tab/Bottom Navigations}.
Eine dem Flow Chart entsprechende Seitenstruktur ermöglicht es, Seiteninhalte isoliert zu bearbeiten.
Die Implementierung der Karte wurde ebenfalls gemeinsam gelöst, allerdings nach Betriebssystemen getrennt.
Sobald der Rahmen der App aufgebaut war arbeiteten wir an unterschiedlichen Funktionalitäten weiter.
\\
\underline{Learnings:}
Bereits in diesem frühen Stadium konnten wir einige Aha-Momente für uns entdecken.
Die Tabs und Bottom Navigation, so wie in unseren MockUps skizziert, ließ sich nur teilweise realisieren.
Eine weitere Recherche ergab, dass der Aufbau der Navigation in React Native eine andere Lösung für diesen Fall vorgab und wir die Implementierung entsprechend anpassen mussten.
Darüber hinaus stießen wir mit dem React Native Modul für die Google Maps Karte (react-native-maps) bereits auf ein undokumentiertes Modul aus der Community, welchem noch weitere folgen sollten.

\subsection{Katharina}
Die Umsetzung der Inhalte des Lexikons und der Lexikon-Detailansicht sowie die Integration des Bewegungssensors übernahm Katharina Garrecht.
Hierbei implementierte sie die im Prototypen definierten Elemente sowohl für die Übersichtsseite als auch für das Template der Kaffee-Detailseite.
Die Detailseiten werden dabei mit in einer JSON Datei hinterlegten Inhalten gespeist.
Auf der Übersichtsseite selbst ist der Bewegungssensor aktiv, welcher bei Erkennung einer Schüttelgeste zufällig eine der zur Verfügung stehenden Kaffee-Zubereitsungsarten auswählt und vorschlägt.
\\
\underline{Learnings:}
Die Integration des Bewegungssensors in die bestehende Navigation brachte einige Stolpersteine mit sich, die nur durch lange Recherchen und Umwege überwunden werden konnten.
Auch die Einarbeitung in native UI Komponenten der jeweiligen Betriebssysteme war für die Umsetzung dieser Funktionalitäten notwendig.

\subsection{Thomas}
Die Darstellung der Listenansicht aller nahen Cafés sowie die Detailansicht einzelner Cafés fiel in den Aufgabenbereich von Thomas Wedler.
Ebenso Teil dieses Bereichs war die Implementierung der Geolocation zur Ermittlung des aktuellen Standorts.
Über diese Geokoordinaten werden bei Bedarf die \emph{Google Places API} und auch die \emph{Google Maps API} angesprochen.
Das Ergebnis ist situationsabhängig entweder eine Liste aller umliegenden Cafés mit rudimentären Informationen zu diesen oder aber ein ausführliches Objekt mit zahlreichen Detailinformationen über ein einzelnes Café.
\\
\underline{Learnings:}
Für die Umsetzung der benötigen Funktionalitäten war eine intensive Einarbeitung in das Ecosystem sowie die bereitgestellten Schnittstellen der Google APIs unumgänglich.
Ebenfalls hier mussten native UI Komponenten der jeweiligen Betriebssysteme korrekt identifiziert und implementiert werden.







??? brauchen wir das? teilweise schon in Setup beschrieben ???

\section{Hands-on}
Praktischer Teil, was lief gut, was nicht

\subsection{Entwicklungsumgebung}
VS Code
Simulatoren
Einbindung in Android Studio \& XCode

!!! Learnings: hier passend, aber anders formulieren !!!
Brücke zwischen Web- und mobiler Entwicklung: Brücke zwischen Web- und mobiler Entwicklung (einfacher Einstieg)
Umständliches Debugging (Simulator -> kein CSS debugging)
Native Abhängigkeiten trotz Cross-Plattform (Libraries einbinden, Environment konfigurieren)
Viele aber unausgereifte Module zur Verfügung – nicht immer ganz ausgereift oder gut dokumentiert

\subsection{Module}
axios: Erleichterung der Kommunikation mit APIs
react-native-maps: Google Maps Karte für Kartenansicht
react-native-permissions: Abfrage von Berechtigungen für Geolocation Zugriff
react-native-sensors: Schnittstelle für Accelerator (Bewegungssensor)
react-native-star-rating: Komponente zur visuellen Darstellung der Gesamtbewertung einzelner Cafés
react-native-vector-icons: Icon Bibliothek
react-navigation: Community Modul für ein alternatives Navigationskonzept

unused:
react-native-material-bottom-navigation: Bottom Navigation im Material Design für Android -> nicht benötigt, da stattdessen Tab Navigation
react-native-stars: Alternative zu react-native-star-rating -> wird nicht benötigt
react-native-google-places-autocomplete: Autocomplete für Google Places in der Nähe -> Nicht genug Funktionalitäten -> react-native-maps

??? eventuell Module auflisten, beschreiben und vor Arbeitsaufteilung packen ???






\section{Vorstellung der App}
??? Beide Plattformen ???
??? Text und Bild nebeneinander ???

Startseite:
Der Einstiegspunkt der App.
Von dort aus ist die Navigation zur Umkreissuche oder zum Lexikon möglich.
(screenshot)
\\
Kartenansicht:
Hier werden auf einer interaktiven Google Maps Karte alle sich in der Nähe befindlichen Cafés einschließlich der aktuelle Standort angezeigt.
Die hierfür benötigen Informationen werden über die entsprechenden Google APIs bezogen.
Ein Klick auf einen Marker öffnet ein \emph{Bottom Sheet}, in welchem die wichtigsten Informationen über das ausgewählte Café wie etwa Entfernung, Bewertung, Öffnungszeiten oder Adresse angezeigt werden.
Innerhalb dieses Sheets stehen Buttons zur Navigation auf die Detailseite des Cafés oder direkt zu Google Maps mit dem Café als hinterlegtem Ziel zur Verfügung.
Des Weiteren erreicht man von hier aus über die Tab Navigation in Android sowie die Bottom Navigation in iOS die Listenansicht der angezeigten Cafés.
(screenshot)
\\
Listenansicht:
Die bereits in der Kartenansicht dargestellten Standorte der Cafés werden hier in einer übersichtlichen Liste gezeigt.
Jeder Listeneintrag enthält die gleichen Informationen und Interaktionsmöglichkeiten wie das Bottom Sheet aus der Karte, da es sich hierbei technisch um eine eigens konstruierte React Native Komponente handelt.
(screenshot)
\\
Infos Café:
Diese Seite zeigt die bereits im Bottom Sheet vorhandenen Informationen über das ausgewählte Café.
Außerdem sind weitere Details wie Telefonnummer, Internetadresse oder auch Bilder des Cafés - soweit über die Google API bereit gestellt - verfügbar.
Auch hier erreicht man über einen Button das Café als Zieladresse direkt in Google Maps.
Darüber hinaus enthält die Tab Navigation respektive Bottom Navigation einen Link zur Bewertungsseite des Cafés.
(screenshot)
\\
Bewertungen Café:
Alle über die Google API verfügbaren Bewertungen des ausgewählten Cafés werden hier in einer Liste angezeigt.
Dabei erhält man Infos wie den Wert der abgegebenen Bewertung, den Zeitraum, den Berwertungsext sowie den Namen des Re­zen­senten.
(screenshot)
\\
Lexikon:
Die Übersichtsseite des Lexikons zeigt alle in der App hinterlegten Zubereitungsarten mit Namen und Icon.
Ein Klick auf eines dieser Elemente führt auf die zugehörige Detailseite.
Außerdem ist dies die einzige Seite, auf der der Bewegungssensor aktiv ist.
Wird eine Schüttelgeste ausgeführt, so erhält man eine zufällig ausgewählte Zubereitungsart in einem Alert angezeigt.
Von diesem aus kann man ebenfalls entweder direkt zur vorgeschlagenen Detailseite gelangen oder den Alert wieder entfernen.
(screenshot)
\\
Lexikon Detailansicht:
Hat man sich für eine Zubereitungsart entschieden erscheint das Template dieser Seite gefüllt mit den passenden Informationen.
Dazu gehören der Name, das Icon, eine Beschreibung des Ablaufs sowie einige Attribute wie Stärke des Getränks oder Menge des benötigten Wassers.
Diese Informationen sind an die jeweils ausgewählte Zubereitungsart angepasst.
(screenshot)



\section{Readme}






\chapter{Ausblick}
\label{ausblick}
...


\section{Testkonzept}

Usability Test
10 – 20 Versuchsteilnehmer aus der Zielgruppe
Beobachtung
Thinking Aloud

Interview und A/B-Test als Teil des Usability-Tests


Interview
Halb-strukturiert
offene Fragen -> allgemein zur App
geschlossene Fragen ->   A/B-Test

A/B-Test
Analyse Schüttelfunktion
Vorschlag Kaffeesorte
Bezahlfunktion



\section{Marketing}
Produktpolitik
App
Web-App
Preispolitik
Kostenfreier Download
Eigenes Café als Top-Café (Anzeige)
Userdaten / In-App Werbung?
Cafédaten
Distributionspolitik
App-Stores
Browser
Kommunikationspolitik
Landing-Page / Social Pages
Werbung in Cafés
Influencer
Social Media Werbung
Zeitung
Blog
Coffee of the Week, Location of the Week, Kaffee-Trends
Kaffee-Reporter (Gesicht für die App)
"Support your Locals" - Communities

Retention Marketing
Gutschein für das Café durch die App
Cafébesitzer müssen zustimmen
Gamification
Punkte bei Besuch neuer Cafés
Top10 bei jeweiligem Café
Auf den Seltenheitsgrad von Kaffeesorten aufmerksam machen
Kaffee-Trends
Lexikoneinträge durch Cafébesuch freischalten
Zeitliche Limitierung

=> Marketing: Eigenes Themengebiet mit vielen Facetten






\chapter{Fazit}
\label{fazit}


\section{Lessons Learned}
Erste Schritte mit React Native und Cross-Plattform-Entwicklung
Umgang mit APIs
Umsetzung der nativen Designs
Ansprechen von Hardwareschnittstellen
Herausforderungen bei der Benutzung weniger etablierter Frameworks
Technische Konzeption: Anforderungen und Implementierung <-> Produktkonzeption (König): Zielgruppe, Marktanlyse, etc.




\section{React Native}
Brücke zwischen Web- und mobiler Entwicklung: Brücke zwischen Web- und mobiler Entwicklung (einfacher Einstieg)
Umständliches Debugging (Simulator -> kein CSS debugging)
Native Abhängigkeiten trotz Cross-Plattform (Libraries einbinden, Environment konfigurieren)
Viele aber unausgereifte Module zur Verfügung – nicht immer ganz ausgereift oder gut dokumentiert


\section{Cross-Plattform vs. Nativ}
Angelehnt an Webentwicklung
Gute Designmöglichkeiten
Hardwarezugriff
Für ComfyCoffee nativ zu aufwändig

=> Cross-Plattform Entwicklung lieber als native Entwicklung