% Seitenstil
\usepackage{fancyhdr}

% Anpassung an Landessprache ---------------------------------------------------
\usepackage[ngerman]{babel}

% Umlaute ----------------------------------------------------------------------
% \usepackage[utf8]{inputenc}
% \usepackage[T1]{fontenc}
% \usepackage{textcomp} % Euro-Zeichen etc.

% Schrift ----------------------------------------------------------------------
\usepackage{lmodern} % bessere Fonts
\usepackage{relsize} % Schriftgr��e relativ festlegen
\newcommand{\mysub}[1]{\raisebox{-0.5ex}{\scriptsize{#1}}}

\usepackage{titlesec}

%Formeln
\usepackage{nicefrac}

% Grafiken ---------------------------------------------------------------------
% Einbinden von JPG-Grafiken erm�glichen
\usepackage[dvips,final]{graphicx}
% hier liegen die Bilder des Dokuments
\graphicspath{{Bilder/}}

% Befehle aus AMSTeX f�r mathematische Symbole z.B. \boldsymbol \mathbb --------
\usepackage{amsmath,amsfonts}

% f�r Index-Ausgabe mit \printindex --------------------------------------------
\usepackage{makeidx}

% Einfache Definition der Zeilenabst�nde und Seitenr�nder etc. -----------------
\usepackage{setspace}
\usepackage{geometry}

% URL verlinken, lange URLs umbrechen etc. -------------------------------------
\usepackage{url}

% PDF-Optionen -----------------------------------------------------------------
\usepackage[
    bookmarks,
    bookmarksopen=true,
    colorlinks=true,
    % diese Farbdefinitionen zeichnen Links im PDF farblich aus
        linkcolor=black, % einfache interne Verknüpfungen
        anchorcolor=black,% Ankertext
        citecolor=blue, % Verweise auf Literaturverzeichniseinträge im Text
        filecolor=magenta, % Verknüpfungen, die lokale Dateien öffnen
        menucolor=red, % Acrobat-Menüpunkte
        urlcolor=blue,
% diese Farbdefinitionen sollten f�r den Druck verwendet werden (alles schwarz)
    %linkcolor=black, % einfache interne Verkn�pfungen
    %anchorcolor=black, % Ankertext
    %citecolor=black, % Verweise auf Literaturverzeichniseintr�ge im Text
    %filecolor=black, % Verkn�pfungen, die lokale Dateien �ffnen
    %menucolor=black, % Acrobat-Men�punkte
    %urlcolor=black,
    %backref,
    plainpages=false, % zur korrekten Erstellung der Bookmarks
    pdfpagelabels, % zur korrekten Erstellung der Bookmarks
    hypertexnames=false, % zur korrekten Erstellung der Bookmarks
    linktocpage % Seitenzahlen anstatt Text im Inhaltsverzeichnis verlinken
]{hyperref}

% Befehle, die Umlaute ausgeben, f�hren zu Fehlern, wenn sie hyperref als Optionen �bergeben werden
\hypersetup{
    pdftitle={\titel},
    pdfauthor={\autor},
    pdfcreator={\autor},
    pdfsubject={\titel},
    pdfkeywords={\titel},
}

% fortlaufendes Durchnummerieren der Fu�noten ----------------------------------
\usepackage{chngcntr}

% f�r lange Tabellen -----------------------------------------------------------
\usepackage{longtable}
\usepackage{colortbl}
\usepackage{array}
\usepackage{ragged2e}
\usepackage{lscape}

% Spaltendefinition rechtsb�ndig mit definierter Breite ------------------------
\newcolumntype{w}[1]{>{\raggedleft\hspace{0pt}}p{#1}}

% Formatierung von Listen �ndern -----------------------------------------------
\usepackage{paralist}

% bei der Definition eigener Befehle ben�tigt
\usepackage{ifthen}

% definiert u.a. die Befehle \todo und \listoftodos
\usepackage{todonotes}

% sorgt daf�r, dass Leerzeichen hinter parameterlosen Makros nicht als Makroendezeichen interpretiert werden
\usepackage{xspace}

% APA-Zitierregeln laden
\usepackage{apacite}

\usepackage{caption}