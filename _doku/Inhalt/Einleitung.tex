\chapter{Einleitung}
\label{einleitung}
Starbucks, wohin das Auge blickt - das bedeutet Massenabfertigung an der Kasse, Kaffee mit Zusatz- und Konservierungsstoffen, Steuervermeidungstaktiken, angeblich ungerecht behandelte Mitarbeiter und unklare Fairness im Kaffeeanbau. Die kleinen, gemütlichen Kaffees verschwinden Aufgrund des hohen Konkurrenzdrucks von billigeren Anbietern immer mehr aus dem Stadtbild. Fair gehandelter Kaffee und exotische Sorten werden immer seltener angeboten.

\section{Ziel}
Unser Bestreben war es daher, kleinen lokalen Cafés wieder mehr Aufmerksamkeit zu schenken. Unsere App \emph{comfycoffee}, die dies umsetzt, verfolgt deshalb vor allem zwei große Ziele:
\begin{itemize}
	\item Finde Geheimtipps für lokale Cafés im Umkreis
	\item Lerne neue Kaffeesorten und Zubereitungsarten kennen
\end{itemize}
Comfycoffee soll auf kleine lokale Cafés aufmerksam machen und im Idealfall deren Kundenstamm erhöhen. Kaffeeliebhaber können die großen Ketten mit ihrem ``0815''-Kaffee umgehen, lokale Cafés entdecken und neue Erfahrungen machen. Zusätzlich ist die App als Lexikon nutzbar, indem sie den Nutzer über Kaffeesorten informiert.

\section{Unique Selling Proposition (USP)}
Der USP von comfycoffee betont die oben genannten Ziele:

\begin{center}
\emph{``Anzeige lokaler Cafés und damit nachhaltige Unterstützung der lokalen Kaffeewirtschaft gegen die großen Ketten''}
\end{center}

Durch das Bekanntmachen kleiner Cafés kann deren Umsatz erhöht werden, so dass sie sich eher gegen große Ketten wie Starbucks und Co. behaupten können.