\chapter{Fazit}
\label{fazit}
Die Konzeption und Umsetzung einer Cross-Plattform-App war für uns Neuland, ebenso wie das Ausarbeiten eines Marketingkonzepts (vgl. Kapitel \ref{sec:marketing}). Im Folgenden sind unsere gewonnenen Erkentnisse und ein Fazit zum Entwicklungsframework React Native sowie der Cross-Plattform-Entwicklung im Vergleich zur nativen Entwicklung aufgeführt.

\section{Lessons Learned}
Comfycoffee war die erste Cross-Plattform App, die wir entwickelt haben. Auch mit dem Framework React Native haben wir unsere ersten Schritte gemacht (vgl. Kapitel \ref{sec:reactnative}). Dabei haben wir den Umgang mit APIs vertieft. Das Umsetzen von nativem Design in Android und iOS entsprechend den jeweiligen Design Guidelines birgt einige Herausforderungen. So haben Android und iOS teilweise die gleiche Bezeichnung für verschiedene Elemente, oder verschiedene Elemente mit dem gleichen Namen. Auch das Ansprechen von Hardwareschnittstellen (GPS, Bewegungssensor) war neu. Zu verstehen, wie diese funktionieren und benutzt werden können bedeutet oft viel lesen und, je nach Aussagekraft der Dokumentationen, noch mehr ausprobieren.

Herausforderungen waren auch gegeben bei der Benutzung eines weniger etablierten Frameworks wie React Native (vgl. Kapitel \ref{sec:reactnativefazit}).

Neu war außerdem die technische Konzeption mit der Implementierung der zuvor ausgearbeiteten, technischen Anforderungen. Die technische Konzeption steht der Produktkonzeption gegenüber, bei der Zielgruppen, Design und Testkontzept ausgearbeitet und Marktanalysen durchgeführt werden. Den teil der Produktkonzeption wurde schon in dem Modul Mobile Usability behandelt. Mit comfycoffee die Verbindung zwischen Produkt- und technischer Konzeption und Umsetzung herzustellen war eine spannende Herausforderung.

\section{React Native}
\label{sec:reactnativefazit}
Das Cross-Plattform Framework React Native schlägt die Brücke zwischen Web- und mobiler Entwicklung. Durch die eingesetzten Webtechnologien wie JavaScript fiel uns durch unsere Affinität zur Webentwicklung der Einstieg nicht schwer. Dennoch hat React Native einige Nachteile bei der Entwicklung. Das Debugging von CSS beispielsweise ist im Simulator nicht möglich. Weiterhin bestehen native Abhängigkeiten trotz der Cross-Plattform Eigenschaften des Frameworks. So mussten Libraries eingebunden und die Environment-Dateien separat per Hand konfiguriert werden. Es stehen online viele Module zur Erweiterung und Einbindung weiterer Elemente in React Native zur Verfügung, jedoch sind diese nicht immer ganz ausgereift oder gut dokumentiert.

\section{Cross-Plattform vs. Nativ}
Die Cross-Plattform Entwicklung mit React Native ist angelehnt an die Webentwicklung. Dies bietet gute Designmöglichkeiten dank CSS- und HTML-ähnlicher Strukturen. Der Hardwarezugriff ist dennoch gegeben und leicht anwendbar. Für eine kleine App wie ComfyCoffee wäre eine native Entwicklung für Android und iOS zu aufwändig gewesen. Die Vorteile der Cross-Platform Entwicklung - die Nutzung von Webtechnologien, die Designmöglichkeiten und der Hardwarezugriff - wiegen vor allem im Fall von comfycoffee die Nachteile auf. Die Entscheidung, womit eine App entwickelt werden soll, hängt zweifelsohne immer von den Anforderungen und Zielen der App ab. Wenn es machbar ist, würden wir jedoch lieber Cross-Platform bzw. hybrid statt nativ entwickeln.