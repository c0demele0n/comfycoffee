% Meta-Informationen -----------------------------------------------------------
%   Definition von globalen Parametern, die im gesamten Dokument verwendet
%   werden k�nnen (z.B auf dem Deckblatt etc.).
%
%   ACHTUNG: Wenn die Texte Umlaute oder ein Esszet enthalten, muss der folgende
%            Befehl bereits an dieser Stelle aktiviert werden:
%            \usepackage[latin1]{inputenc}
% ------------------------------------------------------------------------------
\usepackage[utf8]{inputenc}
\usepackage[T1]{fontenc}
\usepackage{textcomp} % Euro-Zeichen etc.

\newcommand{\logo}{logo.jpg}
\newcommand{\institution}{Hochschule Worms}
\newcommand{\zusatz}{University of Applied Sciences}
\newcommand{\fachbereich}{Informatik}

\newcommand{\art}{Dokumentation}

\newcommand{\titel}{comfycoffee}
\newcommand{\untertitel}{Mobile Web}

\newcommand{\autor}{Thomas Wedler, Katharina Garrecht}
\newcommand{\strasse}{Schraderstr. 36}
\newcommand{\ort}{67227 Frankenthal}
\newcommand{\email}{inf2920@hs-worms.de, inf2930@hs-worms.de}
\newcommand{\studiengang}{Mobile Computing}
\newcommand{\matrikelnummer}{673324, 673342}
\newcommand{\fachsemester}{2}

\newcommand{\betreuer}{Prof. Dr. Elisabeth Heinemann}
\newcommand{\zweitpruefer}{Prof. Dr. XY}

\newcommand{\einreichungsdatum}{31.01.2018}