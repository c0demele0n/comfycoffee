\addchap*{Abstract}

Für die Entwicklung einer App existieren aktuell diverse Entwicklungsstrategien, welche jeweils eigene Vor- und Nachteile für die Applikation selbst mit sich bringen.
Als moderne Strategie gilt hierbei die von Google definierte Progressive Web App (PWA).
Diese webbasierte Art der App erweitert die bereits etablierte Web App um zahlreiche Funktionalitäten, welche durch moderne Webtechnologien realisiert werden können.\\
Als erweiterte Form der Web App versucht die PWA die Nachteile ihrer Vorreiterin auszugleichen.
Hierbei gibt es allerdings keine konkrete Definition einer PWA.
Diese sollte lediglich einige Grundeigenschaften wie Offline-Verfügabarkeit oder sichere Datenübertragung aufweisen und mit zusätzlichen Optimierungen, bspw. im Bereich Performance, angereichert werden.
Doch auch die PWA bringt Nachteile, wie z. B. die aktuell mangelhafte Unterstützung durch mobile Browser, mit sich.\\
Um die PWA auf ihre Zielgruppen sowie relevante Anwendungsbereiche hin zu spezifizieren, werden in dieser Arbeit von Google veröffentlichte Fallstudien qualitativ untersucht.
Dabei werden unterschiedliche Kennzahlen aus diesen Studien miteinander verglichen und etwaige Gemeinsamkeiten herausgearbeitet.
Hauptaugenmerk bei dieser Untersuchung liegt dabei auf der Suche nach Charakteristika zur Zielgruppenbestimmung sowie relevanten Anwendungsbereichen für PWAs im Allgemeinen.\\
Das Ergebnis dieser Ausarbeitung bestärkt die Nutzung einer PWA als Entwicklungsstrategie, da sich durch diese eine Reihe an Erfolgsparametern verbessern lässt.
Hinsichtlich der Untersuchung stellen sich u. a. unzuverlässige Netzwerkverbindungen und begrenzte Datenübertragungsraten als Charakteristika der Zielgruppe für PWAs heraus.
Konkrete Anwendungsbereiche für diese werden alledings nicht definiert, da die Erfolge fallstudienübergreifend unabhängig von unternehmerischen und funktionalen Faktoren erreicht werden.